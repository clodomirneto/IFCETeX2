Apresentação concisa dos pontos relevantes de um documento. O resumo deve apresentar uma visão rápida e clara do conteúdo e das conclusões do trabalho. Deve Informar ao leitor finalidades, metodologia, resultados e conclusões do documento, de tal forma que este possa, inclusive, dispensar a consulta ao original. A primeira frase do resumo deve ser significativa e expressar o tema principal do trabalho. A seguir deve-se indicar a informação sobre a categoria do tratamento (memória, estudo de caso, análise da situação etc.) Deve-se usar o verbo na voz ativa e na terceira pessoa do singular, contendo de 150 a 500 palavras. O resumo deve ser composto de uma sequência de frases concisas, afirmativas e não de enumeração de tópicos. Recomenda-se uso de parágrafo único e justificado, mesma fonte do trabalho, e espaçamento entrelinhas 1,5. Resumo resumo resumo resumo resumo resumo resumo resumo resumo resumo resumo resumo resumo resumo resumo resumo resumo resumo resumo resumo resumo resumo resumo resumo resumo resumo resumo resumo resumo resumo resumo resumo resumo resumo resumo resumo.( ABNT: 6028).

% Separe as palavras-chave por ponto
\palavraschave{Palavra 1. Palavra 2. Palavra 3.}