\chapter{Introdução}

Tem como finalidade dar ao leitor uma visão concisa do tema investigado ressaltando-se: o assunto de forma delimitada, ou seja, de forma que fique evidente sobre o que se está investigando; a justificativa da escolha do tema; a metodologia (como foi feito); os objetivos do trabalho; o objeto de pesquisa que será investigado durante o transcorrer da pesquisa.

Essas informações podem ser feitas em texto corrido (sem subdivisões), porém, se o aluno preferir, poderá, dentro do grande tópico ``Introdução'', fazer subdivisões para: objetivos, metodologia e justificativa.

Todo texto deve ser digitado em fonte Arial ou Times, tamanho 12, inclusive a capa, com exceção das citações com mais de três linhas, notas de rodapé, paginação, dados internacionais de catalogação-na-publicação (ficha catalográfica), legendas e fontes das ilustrações e das tabelas, que devem ser em fonte tamanho menor. O texto deve ser justificado, exceto as referências, no final do trabalho, que devem ser alinhadas à esquerda.

Todos os autores citados devem ter a referência incluída em lista no final no trabalho. (ABNT 14724:2011).